\documentclass[12pt]{article}
\usepackage{amsmath}
\usepackage{pdflscape}
\usepackage{pbox}
\usepackage{framed}
\usepackage{listings}
\lstset{language=Python} 
\usepackage{fancybox}

\makeatletter
\newenvironment{CenteredBox}{% 
\begin{Sbox}}{% Save the content in a box
\end{Sbox}\centerline{\parbox{\wd\@Sbox}{\TheSbox}}}% And output it centered
\makeatother

\makeatletter
\def\BState{\State\hskip-\ALG@thistlm}
\makeatother

\begin{document}
\pagenumbering{gobble}
\begin{landscape}

\centerline{\textbf{The Fibonacci numbers} are important in many aspects of mathematics, computing and are often found in nature}
$$ 
F(n) = \Bigg\{\begin{matrix}
		  0, \textit{ if } n = 0;  \\
		  1, \textit{ if } n = 1;  \\
		  F(n-1) + F(n-2)  \textit{ if } n > 1
		\end{matrix}
 $$
\centerline{\textbf{The Fibonacci numbers} are also intimately connected with the golden ratio, and are typically computed using recursion}
% Source: http://scottsievert.github.io/blog/2015/01/31/the-mysterious-eigenvalue/
\\\\
\centerline{A better way to compute \textbf{The Fibonacci numbers} is done using \textbf{eigenvalues and eigenvectors}}

\begin{figure}[thp]
	\begin{CenteredBox}
	  	\begin{lstlisting}
		def fib(n):
		    lambda1 = (1 + sqrt(5))/2
		    lambda2 = (1 - sqrt(5))/2
		    return (lambda1**n - lambda2**n) / sqrt(5)
		def fib_approx(n)
		    # for practical range, percent error < 10^-6
		    return 1.618034**n / sqrt(5)
		\end{lstlisting}
	\end{CenteredBox}
\caption{Bla}
\end{figure}



\end{landscape}
\end{document}